% !TEX root = ../thesis.tex

\begin{summary}

无人艇是智能化、信息化与传统船舶技术的创新集成。作为能在海上独立执行任务的载体,
麻雀虽小、五脏俱全。由于海洋环境的特殊性,无人艇的技术难度不亚于无人机、无人车,
这也导致海洋无人装备发展滞后于空天、陆地。

无人船艇自主航行,应具备4项基本能力:一是能够通过各类传感器完整地感知水面、水下态势;
二是对各类感知目标进行分类,分析判断环境威胁;三是根据任务要求,规划最优路径和航行计划,
并具备根据环境变化实时调整计划的能力;四是精准控制航向航迹,自主或遥控完成规定任务。

这些能力的形成,需要依托传统的船舶导航、通信、动力、作战系统,加上智能感知、规划决策、
自动控制等方面的最新成果,构成一个完整的从观察到思考,再到执行和反馈的完整链路。

该技术文档是我们团队在无人船研发过程中的技术总结,
包括各类传感器、通信、感知、规划、导航、控制、操作界面等系统。

\end{summary}
